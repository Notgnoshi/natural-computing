\documentclass{article}
\usepackage{algorithm}
\usepackage{algpseudocode}
\usepackage{amsmath}
\usepackage{amssymb}
\usepackage{amsthm}
\usepackage[titletoc]{appendix}
\usepackage{array}
\usepackage[english]{babel}
\usepackage{booktabs}
\usepackage{cancel}
\usepackage{color}
\usepackage{eqparbox}
\usepackage{float}
\usepackage[margin=1in]{geometry}
\usepackage{graphicx}
\usepackage[hidelinks]{hyperref}
% *must* be loaded after hyperref
\usepackage[toc, acronym, numberedsection=nameref]{glossaries}
\usepackage[utf8]{inputenc}
\usepackage{lipsum}
\usepackage{mathtools}
\usepackage[cache=false]{minted}
\usepackage{parskip}
\usepackage{pgfplots}
\usepackage{scalerel}
\usepackage{skull}
\usepackage{subcaption}
\usepackage{titling}
\usepackage{textcomp}
\usepackage{tikz}
\usepackage[compact, explicit]{titlesec}
\usepackage{textcomp}
\usepackage[nottoc]{tocbibind}
\usepackage[textsize=small]{todonotes}
\usepackage[normalem]{ulem}

% Document Settings

\definecolor{__minted_background_color}{rgb}{0.95, 0.95, 0.98}
\definecolor{__minted_highlight_color}{rgb}{0.88, 0.88, 1.0}
\setminted{autogobble=true,
    style=tango,
    breaklines,
    bgcolor=__minted_background_color,
    highlightcolor=__minted_highlight_color,
    mathescape, % Escape math mode everywhere.
    texcomments,  % Enable latex code inside of comments. Useful for referencing equations.
}

\usetikzlibrary{arrows, shapes, positioning}
\pgfplotsset{compat=1.16}
\numberwithin{equation}{section}
% Sets the width of the margin TODO notes
\setlength{\marginparwidth}{0.84in}
\reversemarginpar{}

% hex #184c9a
\definecolor{__glossary_entry_color}{rgb}{0.094, 0.298, 0.604}
\renewcommand{\glstextformat}[1]{\textbf{\textcolor{__glossary_entry_color}{#1}}}

% Add glos: to the beginning of the glossary labels.
\renewcommand*{\glsautoprefix}{glos:}

% All I want is to have comment italicized, but I cant figure out how
% to properly modify the existing \Comment macro.
% \algrenewcomment[1]{\hfill\eqparbox{COMMENT}{\textit{// #1}}}
\algnewcommand{\IComment}[1]{\Comment{\textit{#1}}}
% enable \autoref with algorithms
\newcommand{\algorithmautorefname}{Algorithm}

% TODO: Should this path be relative to the document root or this file?
\graphicspath{{./figures/}}

% Document Definitions

\newcommand{\C}{\mathbb{C}}
\newcommand{\R}{\mathbb{R}}
\newcommand{\Z}{\mathbb{Z}}
\newcommand{\N}{\mathbb{N}}
\renewcommand{\O}{\mathcal{O}}

\theoremstyle{definition}
\newtheorem{defn}{Definition}[section]

\theoremstyle{plain}
\newtheorem{thm}{Theorem}[section]

\renewcommand{\qedsymbol}{$\skull$}

% An inline TODO command. Doesn't play nicely with \todotableofcontents
\newcommand\todoinline[2][]{\todo[inline, caption={TODO}, #1]{
        \begin{minipage}{\textwidth-4pt}#2\end{minipage}}}

% Draw clouds around things. Useful in mathematical proofs.
\newcommand{\cloud}[4][\dots]{
    \raisebox{-0.4\height}{
        \begin{tikzpicture}
            \node [cloud,
                draw,
                cloud puffs=#2,
                cloud ignores aspect,
                minimum height=#3,
                minimum width=#4] {#1};
        \end{tikzpicture}
    }
}

% make each \section a problem.
\titleformat{\section}[runin]{\large\bfseries}{}{0pt}{\titlerule[1.5pt]\newline\vspace*{-4pt}
    Problem\quad\thesection\newline}[\vspace{0.01ex}{\titlerule[1.5pt]}]

% Make autorefs to sections say "Problem x"
\AtBeginDocument{%
\renewcommand{\sectionautorefname}{Problem}
}

% Use \ceil*{} or \floor*{}
\DeclarePairedDelimiter{\ceil}{\lceil}{\rceil}
\DeclarePairedDelimiter{\floor}{\lfloor}{\rfloor}


\title{Homework 1}
\author{Austin Gill}

\begin{document}
\maketitle

\todoinline{
    Any comments applying to the whole homework?
}

\section{}\label{prob:1}

\subsection{Statement}
Compare the effectiveness of the text's iterated hill climbing and simulated annealing algorithms
to find the max of
\[ f(x) = 2^{-2{\left(\frac{(x - 0.1)}{0.9}\right)}^2}{\big(\sin(5\pi x)\big)}^6\]
with $x\in [0,1]$. Use a real valued representation. Include a plot of the function with the
location of the max and a plot of the estimate as a function of the iteration number. How sensitive
are the algorithms to initial values?
\subsection{Method}

\begin{algorithm}
    % \begin{noindent}
    \begin{algorithmic}
        \Function{hill-climbing}{$f$}
            \State{Initialize $x$}
            \While{not done}\IComment{can be convergence or fixed iteration}
                \State{$x' = x + \text{perturbation}$}
                \If{$f(x') < f(x)$}\IComment{this minimizes $f(x)$}
                    \State{$x = x'$}
                \EndIf{}
            \EndWhile{}
            \State\Return{$x$}
        \EndFunction{}
    \end{algorithmic}
    % \end{noindent}
    \caption{The hill climbing algorithm}\label{alg:hill-climbing}
\end{algorithm}

\begin{algorithm}
    % \begin{noindent}
    \begin{algorithmic}
        \Function{iterated-hill-climbing}{$f, n$}
            \State{$solutions = map\big(\Call{hill-climbing}{f}, \{1, \dots, n\}\big)$}\IComment{A trivially parallelizable operation}
            \State\Return{Best solution}
        \EndFunction{}
    \end{algorithmic}
    % \end{noindent}
    \caption{The iterated hill climbing algorithm}\label{alg:iterated-hill-climbing}
\end{algorithm}

\begin{algorithm}
    % \begin{noindent}
    \begin{algorithmic}
        \Function{simulated-annealing}{$f$}
            \State{Initialize $T$}
            \State{Initialize $x$}
            \While{not done}\IComment{prefer convergence over fixed iteration}
                \State{$x' = x + \text{perturbation}$ }
                \If{$\displaystyle rand(0, 1) < e^{\left(f(x) - f(x')\right)\over T}$}
                    \State{$x = x'$}
                \EndIf{}
                \State{Update $T$}
            \EndWhile{}
            \State\Return{$x$}
        \EndFunction{}
    \end{algorithmic}
    % \end{noindent}
    \caption{The simulated annealing algorithm}\label{alg:simulated-annealing}
\end{algorithm}

\begin{algorithm}
    % \begin{noindent}
    \begin{algorithmic}
        \Function{iterated-simulated-annealing}{$f, n$}
            \State{$solutions = map\big(\Call{simulated-annealing}{f}, \{1, \dots, n\}\big)$}\IComment{A trivially parallelizable operation}
            \State\Return{Best solution}
        \EndFunction{}
    \end{algorithmic}
    % \end{noindent}
    \caption{The iterated simulated annealing algorithm}\label{alg:iterated-simulated-annealing}
\end{algorithm}

\todoinline{
    Implement \autoref{alg:simulated-annealing} and plot results.
}

\subsection{Results}
\todoinline{
    \begin{itemize}
        \item Plot the function.
        \item Plot the estimate of the max as a function of the iteration number.
        \item Examine the sensitivity to the initial values.
    \end{itemize}
}

\begin{figure}[h]
    \centering
    % Make sure to run the notebook before this will compile.
    \includegraphics{prob1/figures/prob1-function.pdf}
    \caption{The objective function $f(x)$}\label{fig:prob1:function}
\end{figure}

\section{}\label{prob:2}
\subsection{Statement}
In lecture we addressed the Traveling Salesman Problem using Simulated Annealing. To speed up
convergence and increase the odds of finding the global extremal, it makes sense to try an
evolutionary algorithm. The mutation operator can be adapted from the SA algorithm. Skip
recombination in this problem. Write an evolutionary algorithm to solve the TSP as generated in the
sample program. Compare deterministic and stochastic selection operators.

\subsection{Method}
\todoinline{
    Outline the method(s) used to implement the selection operator.

    Summarize the evolutionary algorithm.
}
\subsection{Experiments}
\todoinline{
    Write an evolutionary algorithm, and steal the mutation operator from the SA algorithm.

    Compare deterministic and stochastic selection operators.
}
\subsubsection{Experiment 1}
\subsubsection{Experiment 2}

\subsection{Results}
\todoinline{
    Compare solutions

    Compare selection operators
}

\section{}\label{prob:3}
\subsection{Statement}
In \autoref{prob:2} we implemented EA code to solve the Traveling Salesman Problem. In this
problem, implement recombination (crossover) in your EA. For this problem you will need to use an
encoding that prevents crossover that creates an invalid candidate. As before, compare
deterministic and stochastic selection operators.

\subsection{Method}
\todoinline{
    Implement crossover.

    Find the right encoding for the problem?
}

\subsection{Experiments}

\subsubsection{Experiment 1}
\subsubsection{Experiment 2}

\subsection{Results}
\todoinline{
    Compare selection operators.
}

\end{document}
